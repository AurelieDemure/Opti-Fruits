\documentclass{article}
\usepackage[utf8]{inputenc}

\title{Charte projet PP2I partie 2}
\usepackage{array}
\author{Manon Lécubin}
\date{17 Octobre 2022}

\begin{document}

\maketitle

\section{Ressources}
-- moyens à mobiliser :
PC personnels et/ou de l'école, postgress4SQL, python3 

\section{Jalons}
\centering
\begin{tabular}{|p{3cm}|p{5cm}|p{2cm}|} 
  \hline
  Jalon & Description & Date \\
  \hline
  Etape 1 : Exigences opérationnelles & Validation de l'idée lors de la soutenance & 22/10/22 \\
  \hline
  Etape 2 : Création de la base de données & Création de la base de données relative aux fruits et légumes disponibles &  \\
  \hline
  Etape 3 : Réalisation de l'interface & Réalisation de l'interface utilisateur & \\
  \hline
  Etape 4 : Réalisation messagerie & 
  Réalisation d'un système de contact entre les utilsateurs (messagerie) pour obtenir plus d'informations sur un produit en vente ou les conditions d'accueil pour une cueillette & \\
  \hline
  Etape 5 : Ajout d'une carte & Ajout d'une carte permettant de repérer où se situent les fruits et légumes & \\
  \hline
  Etape 6 : Ajout de restrictions sur la carte & Ajouter des rayons sur la carte par rapport à la localisation du client, qui augmentent au cours du temps & \\
  \hline
  Etape 7 : Ajout d'une fonctionnalité pour les dons & Ajouter une fonction qui permet au client de donner ses produits dans les derniers jours de consommation possibles du produit & \\ 
  \hline
\end{tabular}

\section{Risques/Opportunités}
Risques : applications utilisant le même concept déjà établies sur le marché
Opportunités : le système de cueillette directement chez le particulier n'a pas encore été utilisé et permettrait de fortement réduire le gâchis
\end{document}
