\documentclass{article}
\usepackage[utf8]{inputenc}

\title{Présentation Application Web PPII}
\author{Pierre AUGUSTE, Aurélie DEMURE, Nicolas FERNANDEZ, Manon LECUBIN}
\date{16 October 2022}

\begin{document}

\maketitle
\section{Présentation de l'application pour les clients}

	Les pommes, c'est bon. C'est encore mieux quand elles viennent de notre jardin. On sait comment elles ont été faites, elles sont bios, locales, et c'est ce qui nous les fait aimer encore plus. Mais quand notre pommier est trop gros, on a trop de pommes, et bien souvent la majorité tombent par terre et pourrissent. De même quand on part en voyage et qu'on ne peut pas les ramasser, elles pourrissent aussi sans qu'on puisse en profiter.

	Mais pourquoi ne pas en faire profiter les autres ? Comment trouver un moyen d'avertir les personnes intéressées de notre surplus de fruits ou légumes ? C'est de là que vient l'idée de notre application web : une interface de rencontre et de partage, pour échanger, donner et recevoir des fruits et légumes locaux, et venir en aide à nos voisins.
	Facilitant l'accès à des produits bios et locaux, cette application permet aux personnes de vendre ou donner ce qu'ils ont cultivé, ou de rechercher des produits près de chez eux. Pour cela rien de plus simple ; il suffit de s'inscrire, renseigner sa localisation, et spécifier nos besoins. Grâce à une carte interactive, on peut facilement avoir accès à ce que proposent nos voisins.

	Envie de vendre ses produits ? Besoin d'aide pour les récoltes ? Une petite annonce et le tour est joué ! Et en y ajoutant des photos, les autres utilisateurs se feront une meilleure idée. Mais si personne ne veut de mes produits ? Pas de panique, si au bout d'un certain temps personne ne se propose, l'application vous donnera la possibilité de faire don à une association.
	Une envie de fruits ou légumes en particulier ? Vous pouvez simplement faire une recherche, et l'application vous retourne les résultats près de chez vous. Quelque soit votre besoin, les propositions tiendront compte de la distance géographique.
	Enfin, lorsque vous trouvez des personnes intéressées, un système de messagerie permet alors d'échanger diverses informations, de discuter, et d'avoir une trace des engagements de chacun.

	Ainsi, cette application web permet aux jardins privés de limiter le gaspillage, mais aussi d'échanger avec d'autres personnes qui partagent vos envies.

\section{Présentation et description de l'application}

Notre projet consiste en une application web permettant de vendre et d'échanger ses fruits et légumes, tout en proposant des cueillettes directement chez le particulier. Pour cela, il y aura possibilité de déposer des annonces de vente de fruits et légumes, de troc ou encore de cueillette. 

Chaque particulier pourra alors rechercher le produit qu'il désire parmi ces annonces proches de chez lui. En effet, une carte interactive permettra de voir la localisation des différents produits, cette localisation ne sera pas précise mais donnée par zone géographique. Les produits seront proposés dans un périmètre donné autour de la localisation du client, périmètre qui augmentera au fil du temps, dans un objectif de limiter la consommation de carburant provoquée par de longs trajets et donc la pollution atmosphérique. 

Un système de messagerie permettra de mettre en contact la personne intéressée et la personne proposant le produit, afin qu'elles puissent convenir d'une date et d'un lieu de rendez-vous. 
La problématique des cambriolages s'est posée lors de la réflexion sur le système de cueillette, cependant chaque particulier ne donnera sa localisation précise uniquement s'il le souhaite via la messagerie et décidera seul s'il accepte qu'une autre personne vienne chez lui sans sa présence. De plus, les cueillettes peuvent également se faire lorsque la personne est présente, par exemple lors d'une journée de télé-travail.

Dans les derniers jours de consommation du produit, s'il n'a pas pu être vendu ou échangé, il sera proposé à l'utilisateur d'en faire don à une association. 


\end{document}
