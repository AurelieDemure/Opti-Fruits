\documentclass{article}
\usepackage[utf8]{inputenc}

\title{Présentation Application Web PPII}
\author{Pierre AUGUSTE, Aurélie DEMURE, Nicolas FERNANDEZ, Manon LECUBIN}
\date{16 October 2022}

\begin{document}

\maketitle

	Les pommes, c'est bon. C'est encore mieux quand elles viennent de notre jardin. On sait comment elles ont été faites, elles sont bios, locales, et c'est ce qui nous les fait aimer encore plus. Mais quand notre pommier est trop gros, on a trop de pommes, et bien souvent la majorité tombent par terre et pourrissent. De même quand on part en voyage et qu'on ne peut pas les ramasser, elles pourrissent aussi sans qu'on puisse en profiter.

	Mais pourquoi ne pas en faire profiter les autres ? Comment trouver un moyen d'avertir les personnes intéressées de notre surplus de fruits ou légumes ? C'est de là que vient l'idée de notre application web : une interface de rencontre et de partage, pour échanger, donner et recevoir des fruits et légumes locaux, et venir en aide à nos voisins.
	Facilitant l'accès à des produits bios et locaux, cette application permet aux personnes de vendre ou donner ce qu'ils ont cultivé, ou de rechercher des produits près de chez eux. Pour cela rien de plus simple ; il suffit de s'inscrire, renseigner sa localisation, et spécifier nos besoins. Grâce à une carte interactive, on peut facilement avoir accès à ce que proposent nos voisins.

	Envie de vendre ses produits ? Besoin d'aide pour les récoltes ? Une petite annonce et le tour est joué ! Et en y ajoutant des photos, les autres utilisateurs se feront une meilleure idée. Mais si personne ne veut de mes produits ? Pas de panique, si au bout d'un certain temps personne ne se propose, l'application vous donnera la possibilité de faire don à une association.
	Une envie de fruits ou légumes en particulier ? Vous pouvez simplement faire une recherche, et l'application vous retourne les résultats près de chez vous. Quelque soit votre besoin, les propositions tiendront compte de la distance géographique.
	Enfin, lorsque vous trouvez des personnes intéressées, un système de messagerie permet alors d'échanger diverses informations, de discuter, et d'avoir une trace des engagements de chacun.

	Ainsi, cette application web permet aux jardins privés de limiter le gaspillage, mais aussi d'échanger avec d'autres personnes qui partagent vos envies.

\end{document}

