\documentclass{article}
\usepackage[utf8]{inputenc}

\title{Etat de l'art PPII}
\author{Aurélie DEMURE}
\date{10 October 2022}

\begin{document}

\maketitle

\section{Problématique}

Selon une étude réalisée en 2016 par ADEME et France Nature Environnement
& Verdicité, 10 millions de tonnes de nourriture sont gâchées chaque
année en France. Les pertes se retrouvent à plusieurs niveaux :
\begin{itemize}
    \item la production à hauteur de 32\%
    \item la transformation des produits à hauteur de 21\%
    \item la distribution à hauteur de 14\%
    \item la consommation chez les professionnels ou les particpuliers 
    pour 33\%
\end{itemize}

Nous nous sommes demandés comment réduire ce gaspillage à la source :
la production.
Celle-ci peut être réalisée par des professionels qui ont déjà des 
méthodes de récolte et des partenariats de vente mais aussi par les
particuliers. Ces derniers peuvent en effet posséder un verger et se
retrouver avec un surplus de fruits ou de légumes comparé à leur 
consommation, voire un simple manque de temps qui les empêche de les
ramasser. 

Nous nous sommes donc penchés sur les jardins des particuliers dans 
le monde rural. Selon le dictionnaire larousse, un jardin privé est un
"terrain, souvent clos, où l'on cultive des légumes, des fleurs, des 
arbres et arbustes fruitiers et d'ornement ou un mélange de ces plantes."
Le site petitsjardiniers.com ajoute que cet environnement domestique
doit être aménagé pour répondre aux besoins de la famille qui l'utilise
et à son mode de vie.

Pour cet état de l'art, nous allons donc étudier les applications ou les
démarches visant à permettre aux particuliers de vendre ou échanger
leurs aliments voire de les faire ramasser.

\section{Application SEEED}
\begin{enumerate}
    \item Idée : réseau social pour le jardin : mise en contact de
    cultivateurs et particuliers pour faire du troc. Possibilité 
    d'échanger par messages et de vendre ses produits.
    \item Fonctionnement : inscription ou connexion via Facebook et
    personnalisation du profil avec le nom et une photo.
    Utilisation de la localisation et calcul d'une distance autour
    pour présenter des jardins proches (disponibles sous forme de carte).
    Profil acheteur qui peut repérer les offres environantes.
    Profil cultivateur qui vend ou propose un troc.
    Les deux ne sont pas incompatibles.
    Mise en place d'une messagerie pour échanger et finaliser les affaires.
    \item Historique : budget de départ pour l'application : 5000€
    
    Très récente : départ du développement début 2021.
    \item Analyse : application novatrice avec un objectif éthique,
    cependant peu de recul sur la fiabilité de la plateforme notamment
    au niveau de l'utilisation et de la gestion des données avec Facebook.
\end{enumerate}
https://seeed-app.fr/ 

\section{Application et site LePotiron} 
\begin{enumerate}
    \item Idée : trocs, dons, ventes de surplus et rencontres car on produit
    souvent plus que ce qu'on peut ou veut consommer.
    \item Fonctionnement : site gratuit où l'on :
    \begin{itemize}
        \item renseigne une ville
        \item choisit des produits parmi ceux disponibles dans notre entourage
        \item remplit un formulaire de mise en contact
    \end{itemize}
        \item Historique : présentée à la première édition française du startupweekend à Paris. Elle existe depuis 2010 mais s'est 
    renouvelé récemment (2017) en raison d'une ancienne équipe indisponible et de problèmes
    techniques liés à l'inscription.
    \item Analyse : premier site de mise en relation des particuliers et petits producteurs sur une base de produits locaux. Il trouve sa raison d'être dans les engagements écologiques des français.
    Mise en place d'un blog (https://www.lepotiblog.com/) dans la continuité
    de leurs valeurs écologiques et humaines.
\end{enumerate}
https://www.lepotiron.fr/
\section{PlantCatching} Plateforme ayant existé une dizaine d'années mais
qui a dû fermer le 6 octobre 2022 du fait de la concurrence de Facebook
(plus assez d'utilisateurs). A la différence des deux précédentes, elle
permettait également le partage de graines, de plantes et de matériaux de
jardinage.

\section{Site Fruiteefy}
\begin{enumerate}
    \item Idée : créer un réseau pour produire et consommer autrement
    avec une carte interactive et des recherches par ville.
    \item Fonctionnement : créer un compte et donner des informations sur
    son jardin si l'on en possède un pour accéder aux ventes de proximité. 
    \item Analyse : s'inscrit dans une dynamique nationale de valorisation
    des circuits courts et de la consommation locale rémunérant les
    producteurs (qui acquiert une certaine visibilité). Est soutenue par
    plusieurs organismes tels que la région Occitanie et la communauté du 
    Coq Vert qui résulte d'un partenariat entre le Ministère de la 
    Transition Ecologique et ADEME.
\end{enumerate}
https://www.fruiteefy.fr/
https://www.bpifrance.fr/nos-actualites/communaute-du-coq-vert-un-engagement-pour-le-climat

\section{Leaf}
\begin{enumerate}
    \item Idée : localiser et acheter des produits frais entre particuliers.
    Solution économique pour les vendeurs et consommation saine pour les
    acheteurs.
    \item Fonctionnement : application gratuite qui permet d'accéder à
    des produits de tout le terroir français. aucun frais sur les
    transactions.
    \item Historique : cette application est née grâce à une fille 
    d'agriculteurs qui ne trouvait pas comment s'approvisionner en 
    produits frais sans devoir retourner chez ses parents.
    S'inscrit aujourd'hui dans le Plan Alimentaire Territorial.
    \item Analyse : les petits producteurs ne sont pas inclus dans le 
    programme et la consommation locale n'est pas valorisée.
\end{enumerate}
https://www.leafapp.fr/

\section{Initiative Aux Arbres Citoyens}
\begin{enumerate}
    \item Idée : effectuer des cueillettes collectives et solidaires
    chez des propriétaires d'arbres fruitiers pour les redistribuer
    aux plus démunis.
    \item Fonctionnement : les bénévoles et propriétaires peuvent récupérer
    ce qu'ils souhaitent de la récolte mais dans les faits les dons sont
    de l'ordre de 80\%.
    \item Historique : fondée en juillet 2020 à la Rochelle sur trois
    valeurs: la lutte contre le gaspillage
             - les circuits courts
             - la solidarité
    \item Analyse : initiative contre le gaspillage qui est réalisée par
    des bénévoles mais ne s'insert dans aucune application.
\end{enumerate}

\section{Conclusion}

Des applications réussissent déjà à mettre en lien des particuliers pour
le troc ou la vente de fruits et légumes. La plupart sont florissantes même 
si la concurrence semble importante.

Cependant aucune de ces applications ne semblent proposer un équivalent à l'initiative Aux Arbres Citoyens.


De plus les invendus de l'application pourraient être proposés à des 
associations à l'approche de leur date de péremption.


Nous allons donc chercher à insérer ces deux idées dans notre application
pour proposer aux consommateurs et producteurs un modèle plus complet.
\end{document}
