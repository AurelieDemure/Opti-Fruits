\documentclass{article}
\usepackage[utf8]{inputenc}
\usepackage[margin=1.15in]{geometry}
\usepackage{graphicx}
\graphicspath{ {D:/Nicolas/Images/} }
\usepackage{nicematrix}
\usepackage{layout}
\usepackage{hhline}
\usepackage{array}

\newcolumntype{C}[1]{>{\centering\let\newline\\\arraybackslash\hspace{0pt}}m{#1}}

\title{Charte projet}
\author{Nicolas Fernandez}
\date{October 2022}

\begin{document}

\maketitle

\section{Contexte}

Aujourd'hui, les circuits courts sont les alternatives les plus intéressantes face au défi qu'est devenu la gestion de l'alimentation et de l'eau. De nombreuses solutions sont apparues pour faire face à ce défi. Parmi ces solutions, on retrouve les jardins partagés, les micro-fermes mais également le partage de ressources produites au sein de jardins privés. MENTION ETAT DE L'ART.

Ces applications sont principalement utilisées par les propriétaires de jardin privés. REPRENDRE PHRASE CONCLUSION ETAT DE L'ART

Le cahier des charges permet d'identifier de manière immédiate les principaux besoins et fonctions auxquels l'application doit répondre. Il s'agit notamment de mettre en relation des propriétaires de jardins privés, de les aider en participant à la récolte des fruits et enfin de permettre de gérer les surplus de fruits qui pourraient finir à la poubelle.

\maketitle

\section{Finalités et importance du projet: Business Case}

Ce projet est aujourd'hui essentiel afin de lutter face aux problématiques actuelles de sociétés concernant la gestion de l'alimentation en circuits courts. Sa priorité est ainsi justifiée par cette nécessité. De plus, la nature même du projet et ses coûts extrêmements limités réduisent l'impact de l'évaluation de son retour sur investissement, permettant ainsi un départ peu risqué. L'application visant à proposer un service sans rechercche de bénéfices, l'estimation de ces derniers est immédiate.

De manière prévisionnelles, on peut estimer que l'application permettra de mettre en relation de nombreux propriétaires de jardin privés au sein de communes principalement en zones rurales. Sa gestion des récoltes participera à une nouvelle manière de partager des surplus de fruits non consommés, une innovation qui permettra une ouverture plus grande que la plupart des projets existants.

\vspace{4mm}

Les parties prenantes relatives au projets sont: l'équipe projet, composée de quatre membres, les enseignants encadrants, ainsi que les potentiels utilisateurs de l'application.

Le périmètre du projet s'étend sur tout le territoire, mais la nature même de son fonctionnement favorise une utilisation dans des zones rurales où le nombre de jardins privées est bien plus important.
En dehors de cette limite géographique, de nombreux prérequis sont à obtenir: obtention des compétences nécessaires au développement d'une application web et de la création d'algorithme, cadre légal relatif à la possibilité de venir récolter les fruits chez une autre personne.

\vspace{4mm}
 
Cette analyse du Business Case permet de proposer une matrice SWOT identifiant les forces, faiblesses, opportunités ainsi que menaces du projet.

\vspace{2mm}             
  
\begin{centering}
\includegraphics[width = 12cm, height = 9 cm]{SWOT}
\end{centering}

\maketitle
\section{Objectifs et résultats opérationnels}


FAIRE LISTE DES LIVRABLES AVEC SOLUTION OU MOYEN DE LA TROUVER +
CRITERES ET INDICATEURS DE SUCCES


\end{document}



