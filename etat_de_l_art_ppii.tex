\documentclass{article}
\usepackage[utf8]{inputenc}

\title{Etat de l'art PPII}
\author{Aurélie DEMURE}
\date{10 October 2022}

\begin{document}

\maketitle

\section{Application SEEED}
\begin{itemize}
    \item Réseau social pour le jardin : mise en contacts de cultivateurs
    et particuliers pour faire du troc. 
    \item Utilisation de la localisation et calcul d'une distance autour.
    \item Possibilité d'échanger par messages et de vendre ses produits.
    \item Budget de départ pour l'application : 5000€
    \item Très récente.
\end{itemize}

\section{PlantCatching}

\section{LePotiron} 
\begin{itemize}
    \item Site avec une carte pour localiser un producteur en fonction de 
    notre secteur et de l'aliment recherché.
\end{itemize}

\section{Fruiteefy}

\section{Leaf}

\section{Geev alimentaire}

Idée : observer les commmentaires des applis pour se faire une idée des 
forces et faiblesses de chacune ainsi que leurs spécificités.
\end{document}
